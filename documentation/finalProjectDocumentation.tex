\documentclass[12pt]{article}
\usepackage[letterpaper]{geometry}
\usepackage{amsmath, amsthm, amssymb, amsfonts}
\usepackage{graphicx}
\usepackage{titling}
\usepackage{hyperref}
\newcommand{\subtitle}[1]{%
  \posttitle{%
    \par\end{center}
    \begin{center}\large#1\end{center}
    \vskip0.5em}%
}
\hypersetup{
    colorlinks=true,
    linkcolor=blue,
    filecolor=magenta,      
    urlcolor=blue,
} 
\urlstyle{same}
\title{CSE 180 - Robotics \\ Final Project}
\subtitle{Team Alpha}
\author{Christine Breckenridge, Aleksandr Brodskiy, Carlos Martinez}
\begin{document}
\maketitle
{\setlength{\parindent}{0cm}
\textbf{Outline}
\begin{enumerate}  
\item Documentation
\item Design Decisions
\item Testing/QA
\item Design Questions
\item Team Member Work-Log
\item Conclusion\\\\\\\\\\\\\\\\\\\\\\\\\\\\
\end{enumerate} 
}
{\setlength{\parindent}{0cm}
\textbf{Documentation} \\
\paragraph{} In order to produce a complete and sound solution to the task of detecting treasures along an arbitrary map, an approach was devised such that the computations performed by the robotic alogirthms would be distributed.
  In this manner, the initial implementation of this approach involved the construction of two nodes, \textbf{\textit{nav.cpp}} and \textbf{\textit{treasures.cpp}}. Therefore the two nodes would generate functionality in tandem with each other.
  \begin{center} \textit{further improvements are specified in the} \textbf{Design Decisions} \textit{section}. \end{center}
  \paragraph{} The primary data structures utilized in the formulization of this distributed approach were a one dimensional array as well as a map with a \textbf{\textit{string}} type as the key and a \textbf\textit{{pose}} object as the value.
  The simplification of storing a \textbf{\textit{pose}} object within the map was predominantly related to its dependance on the ROS structs \textit{\textbf{point}} and \textit{\textbf{quaternion}} for the computing of the translation and rotation of the robot, respectively.
  The instantiation of the map constructor is demonstrated below with the \textbf{\textit{treasures}} label:
  \begin{center} \textsc{std\textbf{::}map< char[], Pose > \textbf{\textit{treasures}}} \end{center}  
  \paragraph{} Furthermore, by subscribing to the \textit{map} topic it becomes possible to retrieve data embedded in the ROS message:
  \begin{center} \textsc{nav\_msgs/OccupancyGrid} \end{center}
  which contains the \textsc{Header header}, \textsc{MapMetaData info} , and \textsc{int}8[] \textsc{data} fields. The \textsc{MapMetaData info} field provides information regarding the height, width, and resolution of the environment map.
  The height, width, and resolution are defined as follows:
  \begin{center} \textsc{uint}32[] \textsc{height} \\ \textsc{uint}32[] \textsc{width} \\ \textsc{float} 32 \textsc{resolution} \end{center}
  With these variables, objects, and data structures . . . 
  \\\\
}
{\setlength{\parindent}{0cm}
\textbf{Design Decisions} 
\paragraph{} \textit{to be completed upon final submission of the source code} . . . \\\\
}
{\setlength{\parindent}{0cm}
\textbf{Testing/QA}\\
\begin{center}Treasures\end{center}
\paragraph{} \textit{to be completed upon final submission of the source code} . . . \\\\
\begin{center}Navigation\end{center}
\paragraph{} \textit{to be completed upon final submission of the source code} . . . \\\\
}
{\setlength{\parindent}{0cm}
\textbf{Team Member Work-Log}\\ \\
\textbf{\begin{center}\underline{Christine Breckenridge}\end{center}} 
\begin{itemize}
\item Treasures
\end{itemize}
\textbf{\begin{center}\underline{Aleksandr Brodskiy}\end{center}}  
\begin{itemize}
\item Documentation 
\end{itemize}
\textbf{\begin{center}\underline{Carlos Martinez}\end{center}} 
\begin{itemize}
\item Navigation
\end{itemize}
}
{\setlength{\parindent}{0cm}
\textbf{\\\\\\Conclusion}
\paragraph{} \textit{to be completed upon final submission of the source code} . . .
\end{document} 
